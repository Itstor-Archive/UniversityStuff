\documentclass[]{article}
\usepackage{xcolor}
\usepackage{listings}
\usepackage{showexpl}
\usepackage[bahasai]{babel}
\lstset{language=C++,
	% numbers=left,
	%	stepnumber=1,
	numberstyle=\ttfamily,
	basicstyle=\ttfamily,
	keywordstyle=\color{blue}\ttfamily,
	stringstyle=\color{red}\ttfamily,
	commentstyle=\color{gray}\ttfamily,
	morecomment=[l][\color{magenta}]{\#}
}


%opening
\title{Tugas 1: Implementasi Operasi Dasar Pada Array}
\author{Akhmad Thoriq Afif NRP 5024201028}

\begin{document}
\maketitle
\section{Listing Program}
Berikut ini merupakan source code dari tugas 1 Implementasi Operasi Dasar pada Array. Program ini ditulis menggunakan bahasa C++
\lstinputlisting[label={kodingan},caption={Operasi dasar Dalam Array}, language={C++}]{tugas.cpp}
\subsection*{OUTPUT PROGRAM:}
\fbox{\begin{minipage}{40em}
	Mencetak Array definedArray: \newline
	[100.2, 237.1, 127.24] \newline
	Menambahkan 81, 83, 72, 100 ke dalam emptyArray: \newline
	[81, 83, 72, 10] \newline
	Menghapus element dengan index 1 pada emptyArray: \newline
	[81, 72, 10] \newline
	Memasukkan angka 100 index 2 pada emptyArray: \newline
	[81, 72, 100, 10] \newline
	Index angka 100 pada emptyArray: \newline
	2 \newline
	Angka maksimum pada emptyArray: \newline
	100 \newline
	Menghapus angka maksimum pada emptyArray: \newline
	[81, 72, 10] \newline
	Menghapus seluruh isi emptyArray: \newline
	[] \newline
\end{minipage}}
\pagebreak
\section{Penjelasan Program}
\subsection{Fungsi untuk menghapus semua element pada array}
\lstinputlisting[label={hapussemua},caption={Fungsi untuk menghapus seluruh isi Array}, language={C++}, firstline=92, lastline=98]{tugas.cpp}
\par
Dengan memanfaatkan operasi dasar dalam array, dapat dibuat suatu fungsi untuk menghapus seluruh element dalam suatu array. Penghapusan tersebut dilakukan dengan cara memanggil fungsi remove() dengan parameter index 0 secara berulang kali sampai ukuran array tersebut lebih kecil dari pada 0.
\par
Fungsi remove sendiri bekerja dengan cara memindahkan element yang ada pada n+1 ke n, dimana n adalah index dari element yang akan dihapus.
\subsection{Fungsi untuk mencari posisi nilai tertinggi}
\lstinputlisting[label={maxvalue},caption={Fungsi untuk mencari nilai tertinggi}, language={C++}, firstline=100, lastline=115]{tugas.cpp}
\par
Dibutuhkan satu parameter dalam fungsi ini yaitu, ret\_index yang berupa boolean. Paramter ini digunakan untuk menentukan keluaran dari fungsi ini, apakah dalam bentuk index atau nilai maksimum.
\par
Cara kerja dari fungsi ini adalah sebagai berikut:
\begin{enumerate}
	\item Menetapkan element pertama sebagai nilai maksimum dengan menyimpannya ke dalam variabel max.
	\item Melakukan pengaksesan dari element kedua sampai dengan element terakhir.
	\item Membandingkan  variabel max dengan elemeent yang sedang diakses.
	\item Jika, variabel max lebih kecil daripada element yang diakses maka element yang sedang diakses dimasukkan ke dalam variabel max.
	\item Setelah semua element dalam array terakses, dilakukan pengembalian nilai. Jika ret\_index bernilai false maka fungsi akan mereturn variabel max. Jika bernilai true, maka akan dilakukan pemanggilan fungsi search untuk menentukan posisi nilai maksimum tersebut.
\end{enumerate}
\par
Fungsi ini masih bisa dibuat menjadi lebih efisien yaitu dengan membuat variabel baru untuk menyimpan nilai dari variabel i ketika persyaratan max $<$ mArray[i] bernilai true. Variabel inilah yang kemudian akan dikembalikan. Jika cara tersebut diimplementasikan, maka fungsi akan menjadi seperti sebagai berikut:
\lstinputlisting[label={findmaxefc},caption={Fungsi untuk mencari nilai tertinggi}, language={C++}]{findmax.cpp}
\subsection{Fungsi untuk menghapus nilai maksimum}
\lstinputlisting[label={hapusmaks},caption={Fungsi untuk menghapus nilai maksimum}, language={C++}, firstline=117, lastline=120]{tugas.cpp}
\par
Fungsi ini didgunakan untuk menghapus nilai maksimum dalam suatu array. Cara kerja dari fungsi ini memanfaatkan fungsi remove() dan findMaxValue(). 
\end{document}
